\documentclass[a4paper,c5size,twoside,UTF8]{ctexart} %ctexart文档类,A4纸,正文5号字
\RequirePackage{xeCJK}
\RequirePackage{graphicx}
\RequirePackage{geometry}%页面设置
%\RequirePackage{titlesec}%设置标题
\RequirePackage{fancyhdr}%定制页眉页脚
\RequirePackage{enumitem}%定制enumerate环境参数
\usepackage{caption,subcaption}
\usepackage{multirow}%横跨两行以上排版
\usepackage{tabularx} 
\usepackage{amsmath}
\usepackage{makecell}%单独控制表项单元,例如表内使用\\换行
%引用超链接包
\usepackage[dvipsnames]{xcolor}
\usepackage{listings}
\usepackage[colorlinks=true,linkcolor=red,anchorcolor=blue,citecolor=green,bookmarksopen=true]{hyperref}
% =============================================
% Part 1 编辑页面宏观信息(行距、页眉、页脚等)
% =============================================
\linespread{1.5} \selectfont
\geometry{top=1in,bottom=1in,left=1in,right=1in}  %页面设置
%\setcounter{secnumdepth}{2}

%页眉页脚设置
\pagestyle{fancy} %页眉
\fancyhf{}
\fancyfoot[RO]{\thepage}   %奇数右页脚
\fancyfoot[LE]{\thepage}  %偶数左页脚
\fancyhead[RO]{\slshape \leftmark} %奇数右页眉
\fancyhead[LE]{\slshape \leftmark}  %偶数左页眉

%设置插入代码的样式
\lstset{
	language=Mathematica, % 设置语言
	basicstyle=\ttfamily, % 设置字体族
	breaklines=true, % 自动换行
	keywordstyle=\bfseries\color{NavyBlue}, % 设置关键字为粗体,颜色为 NavyBlue
	morekeywords={}, % 设置更多的关键字,用逗号分隔
	emph={}, % 指定强调词,如果有多个,用逗号隔开
	emphstyle=\bfseries\color{Rhodamine}, % 强调词样式设置
	commentstyle=\color{blue}, % 设置注释样式,斜体,浅灰色
	stringstyle=\bfseries\color{gray}, % 设置字符串样式
	columns=flexible,
	numbers=left, % 显示行号在左边
	numbersep=2em, % 设置行号的具体位置
	numberstyle=\footnotesize, % 缩小行号
	frame=single, % 边框
	framesep=1em % 设置代码与边框的距离
}

\begin{document}

% =============================================
% Part 2 主文件
% =============================================

%设置section,subsection,subsubsection格式

%这是一种修改
%\CTEXsetup[format={\bfseries\zihao{4}}]{section}  % section四号字体
%\CTEXsetup[number={\chinese{section}}]{section}
%\CTEXsetup[name={,、} ]{section}
\CTEXsetup[format={\bfseries\zihao{-4}} ]{subsection}%subsection小四号字体
%\renewcommand\thesection{\zhnum{section}}  %section中文编号
\renewcommand\thesubsection{}  %subsection不编号

%这是另一种修改方式,推荐用这种
\ctexset{
	% 修改 section。
	section={   
		name={,、},
		number={\chinese{section}},
		format=\heiti\raggedright\zihao{4}, % 设置 section 标题为黑体、右对齐、4号字
		aftername=\hspace{0pt}
	},
	% 修改 subsubsection。
	subsubsection={   
		name={,.},
		number={\arabic{subsubsection}},
		format=\indent\heiti\raggedright\zihao{5}, % 设置 subsection 标题为黑体、5号字
		aftername=\hspace{0pt}
	}
}

\title{这是实验标题!!!!\vspace{-2em}}
\date{}
\maketitle
\thispagestyle{empty}
%\tableofcontents   %目录

\centerline{学生姓名:姓名 \hfill 学号:U202200001 \hfill 班级:班级名称}

\section{引言}

\subsection{【实验背景】}
这里是实验背景。这里是实验背景。这里是实验背景。这里是实验背景。这里是实验背景。这里是实验背景。这里是实验背景。这里是实验背景。这里是实验背景。这里是实验背景。这里是实验背景。这里是实验背景。这里是实验背景。这里是实验背景。这里是实验背景。这里是实验背景。这里是实验背景。这里是实验背景。这里是实验背景。

\subsection{【实验目的】}
\begin{enumerate}[itemsep=0pt,parsep=0pt]
	\item 实验目的1;
	\item 实验目的2;
	\item 实验目的3;
	\item 实验目的4。
\end{enumerate}

\section{实验内容与数据处理}
%下面是若干可能用到的代码模板!

%这是展示一张图片的代码格式
%	\begin{figure}[h]
	%		\centering
	%		\includegraphics[width=0.7\linewidth]{../图像/电路原理图}
	%		\caption{}
	%		\label{}
	%	\end{figure}

%这是展示两张不相关图片的代码格式
%	\begin{figure}[h]
	%		\begin{minipage}[b]{0.49\textwidth}
		%			\centering
		%			\includegraphics[width=0.7\linewidth]{../图像/前端电路示意图}
		%			\caption{}
		%			\label{}
		%		\end{minipage}
	%		\begin{minipage}[b]{0.49\textwidth}
		%			\centering
		%		    \includegraphics[width=0.7\linewidth]{../图像/交流放大}
		%			\caption{}
		%			\label{}
		%		\end{minipage}
	%	\end{figure}

%这是展示两张相关图片的代码格式
%	\begin{figure}[h]
%		\centering
%		\begin{subfigure}[b]{0.49\textwidth}
%			\centering
%			\includegraphics[width=0.7\linewidth]{../图像/failed}
%			\caption{小标题}
%			\label{小标签}
%		\end{subfigure}
%		\begin{subfigure}[b]{0.49\textwidth}
%			\centering
%			\includegraphics[width=0.7\linewidth]{../图像/交放异常(C小)}
%			\caption{小标题}
%			\label{小标签}
%		\end{subfigure}
%		\caption{大标题}
%		\label{大标签}
%	\end{figure}

%这是某一Mathematica代码格式,为了方便,我没有删除里面的内容。
%	\begin{lstlisting}[mathescape]
%		(*电容C根据实际有的电容值,取XXpF,R4尽可能小,所以取XX$\color{blue}\Omega$,由此计算R3与R5和品质因数Q*)
%		R4 =;(*输入所取的R4值*)
%		c =;(*输入所取的电容C值*)
%		a = NSolve[$\mathtt{R4=\frac{R3}{2\times3\times \left(50\times 10^3\times 2 \pi \right)^2\times c^2\times R3^2-1}}$, R3];
%		R3 = R3 /. a[[1]];
%		R5 = 6*R3;
%		$\mathtt{Q=\sqrt{\frac{3}{2} \left(\frac{R3}{R4}-1\right)}}$;
%		Print["R3=", R3, "$\color{gray}\Omega$"]
%		Print["R5=", R5, "$\color{gray}\Omega$"]
%		Print["品质因数Q=", Q]
%		Clear[R4, c, R3, a, R5, Q]
%	\end{lstlisting}

%这是表格的代码模板。
%	\begin{table}[h]
%		\caption{}
%		\label{}
%		\begin{tabular}{}
%		%这里请使用向导生存
%		\end{tabular}
%	\end{table}

\subsection{【实验原理】}
\subsubsection{第一个实验原理}
这是第一个实验原理。这是第一个实验原理。这是第一个实验原理。这是第一个实验原理。这是第一个实验原理。这是第一个实验原理。这是第一个实验原理。这是第一个实验原理。这是第一个实验原理。这是第一个实验原理。这是第一个实验原理。这是第一个实验原理。这是第一个实验原理。这是第一个实验原理。这是第一个实验原理。这是第一个实验原理。这是第一个实验原理。这是第一个实验原理。这是第一个实验原理。这是第一个实验原理。这是第一个实验原理。这是第一个实验原理。这是第一个实验原理。这是第一个实验原理。这是第一个实验原理。这是第一个实验原理。这是第一个实验原理。这是第一个实验原理。这是第一个实验原理。这是第一个实验原理。

\subsubsection{第二个实验原理}
这是第二个实验原理。这是第二个实验原理。这是第二个实验原理。这是第二个实验原理。这是第二个实验原理。这是第二个实验原理。这是第二个实验原理。这是第二个实验原理。这是第二个实验原理。这是第二个实验原理。这是第二个实验原理。这是第二个实验原理。这是第二个实验原理。这是第二个实验原理。这是第二个实验原理。这是第二个实验原理。这是第二个实验原理。这是第二个实验原理。这是第二个实验原理。这是第二个实验原理。这是第二个实验原理。这是第二个实验原理。这是第二个实验原理。这是第二个实验原理。这是第二个实验原理。这是第二个实验原理。这是第二个实验原理。这是第二个实验原理。这是第二个实验原理。这是第二个实验原理。

\subsection{【实验内容记数据处理和分析】}
\subsubsection{第一个实验内容和数据处理与分析}
第一个实验内容和数据处理分析。第一个实验内容和数据处理分析。第一个实验内容和数据处理分析。第一个实验内容和数据处理分析。第一个实验内容和数据处理分析。第一个实验内容和数据处理分析。第一个实验内容和数据处理分析。第一个实验内容和数据处理分析。第一个实验内容和数据处理分析。第一个实验内容和数据处理分析。第一个实验内容和数据处理分析。第一个实验内容和数据处理分析。第一个实验内容和数据处理分析。第一个实验内容和数据处理分析。第一个实验内容和数据处理分析。第一个实验内容和数据处理分析。第一个实验内容和数据处理分析。第一个实验内容和数据处理分析。第一个实验内容和数据处理分析。第一个实验内容和数据处理分析。

\subsubsection{第二个实验内容和数据处理与分析}
第二个实验内容和数据处理分析。第二个实验内容和数据处理分析。第二个实验内容和数据处理分析。第二个实验内容和数据处理分析。第二个实验内容和数据处理分析。第二个实验内容和数据处理分析。第二个实验内容和数据处理分析。第二个实验内容和数据处理分析。第二个实验内容和数据处理分析。第二个实验内容和数据处理分析。第二个实验内容和数据处理分析。第二个实验内容和数据处理分析。第二个实验内容和数据处理分析。第二个实验内容和数据处理分析。第二个实验内容和数据处理分析。第二个实验内容和数据处理分析。第二个实验内容和数据处理分析。第二个实验内容和数据处理分析。第二个实验内容和数据处理分析。第二个实验内容和数据处理分析。


\subsection{【实验方法和技术】}
这是实验用到的方法和技术。这是实验用到的方法和技术。这是实验用到的方法和技术。这是实验用到的方法和技术。这是实验用到的方法和技术。这是实验用到的方法和技术。这是实验用到的方法和技术。这是实验用到的方法和技术。这是实验用到的方法和技术。这是实验用到的方法和技术。这是实验用到的方法和技术。这是实验用到的方法和技术。这是实验用到的方法和技术。这是实验用到的方法和技术。这是实验用到的方法和技术。这是实验用到的方法和技术。

\subsection{【实验遇到的问题及解决的方法】}
这是实验遇到的问题及解决的方法。这是实验遇到的问题及解决的方法。这是实验遇到的问题及解决的方法。这是实验遇到的问题及解决的方法。这是实验遇到的问题及解决的方法。这是实验遇到的问题及解决的方法。这是实验遇到的问题及解决的方法。这是实验遇到的问题及解决的方法。这是实验遇到的问题及解决的方法。这是实验遇到的问题及解决的方法。这是实验遇到的问题及解决的方法。这是实验遇到的问题及解决的方法。这是实验遇到的问题及解决的方法。


\section{实验小结}
\subsection{【体会或收获】}
通过这个实验,我……

\subsection{【实验建议】}
\begin{enumerate}[itemsep=0pt,parsep=0pt]
	\item  实验建议1;
	\item  实验建议2。		
\end{enumerate}

\begin{thebibliography}{1}
	%参考文献并没有使用.bib文件
  \bibitem{liu} 刘海洋, \LaTeX 入门 [M], 电子工业出版社
  \bibitem{hu}  胡伟,\LaTeX 2e完全学习手册(第二版),清华大学出版社
\end{thebibliography}

		
\section*{附录A:}
%重新编号设置
\setcounter{equation}{0}
\renewcommand{\theequation}{A.\arabic{equation}}
这是附录A。公式重新编号,如:
\begin{equation}
	1+1=2
\end{equation}
\end{document}
